% !TEX encoding = UTF-8 Unicode

\documentclass[twocolumn,10pt,a4j]{jsarticle}
\usepackage{kougai}

\title{論理的文章を支援するツールの開発}
\author{1632144 三浦 恋  指導教員 須田 宇宙 准教授}
\date{}

\begin{document}

\maketitle

\section{はじめに}

%背景
大学生に論理的な思考力や論理的文章作成能力が求められており,大学では論文やレポート書き方,言葉遣いなどを教える初年次教育やレポートの書き方の指導や修正を行うライティングセンターの設置などが進められている.また,2022年には高校国語に論理国語が導入される\cite{ren00}.
 
 
%問題点
しかし初年次教育の授業は論文やレポートの書き方を教えるための授業ではなく,大学で学ぶにつれて必要な知識や技術の基本を身につけることが目的とされている.またライティングセンターが設置されているとしても自発的にライティングセンターを利用しなければ文章作成力は向上しない.そのため実際に論文やレポートを書く際にアウトラインなどの事前準備をせずに文章の作成を行ってしまう学生が多く,主張が一貫した文章にならないことが問題点として挙げられる.

%目的
実際に論文や小説などの長文を作成する際にアウトラインプロセッサが利用されている.多くのアウトラインプロセッサのは作成した文章に見出しをつけ,階層構造で管理することが目的として利用されていため,学生が書く論文やレポートに利用しづらいことが問題点となっている.

そこで本研究では,主張や根拠などが明確な一貫した文章を書く力を身に付け,論文やレポートの作成を支援するアカデミックアウトラインツールを開発することを目的としている.

\section{アカデミックライティングについて}
 アカデミックライティングとは大学で作成が求められる,レポートやレジュメをはじめとし,卒業論文や研究学術論文など学術的な文章を書く技術,書く行為,または書いた物のことをアカデミックライティングという\cite{ren01}と書かれている.また小説や感想文等と異なり,いくつかの特徴が存在する.その中でも重要な特徴を以下に挙げる.
\begin{description}
  \item[(1)] 問いと答えの構造と論理的な説明での構成されている.
  \item[(2)] 主張と根拠が明示されている.
  \item[(3)] パラグラフ構造になっている.
  \item[(4)] 引用の倫理のルールに従っている.
  \item[(5)] 学術的文章に特有の一定の形式に従ってる.
\end{description}

\section{開発したツールについて}
レポートや論文を書く際にアウトラインを作成してから文章を作成する必要があるが実際はアウトラインを作成せずに文章書き始めてしまっている場合が多く,主張が一貫した文章がを作成できていない.本研究では主張が一貫した論理的文章を作成するためにアカデミックライティングに沿ったアウトラインツールを開発する.既存のアウトラインプロセッサと異なり,アウトラインから文章を作成するのではなく,アウトラインを作成する過程を支援する.
具体的にはアカデミックライティングに沿った文章作成をするために必要な技術や情報を集めること,主張や文章構成を整理する機能である.支援するものは2章で述べた(1)と(2)である.


実装する機能として\textcircled{\scriptsize{1}}は「問いと答え」,\textcircled{\scriptsize{2}}は「論理的な構成」,\textcircled{\scriptsize{3}}は「主張と根拠」とする.\textcircled{\scriptsize{1}}「問いと答え」では課題や主張への疑問とそれに対する答えを短い文章で記述し,見返した際に主張からずれた意見が出ることを防ぐことができると考え,課題や主張への疑問とそれに対する答えを短い文章で記述する.\textcircled{\scriptsize{2}}「論理的な構成」では論理的な文章を書く上で書かれた短い文章は問いと答えがセットであるだけであって順番書かれているとは限らないことや序論,本論,結論のどの部分の情報なのかを整理することで論理的な構成を支援するため文章の順番を自由に変更できる機能,序論,本論,結論のどの部分の情報なのか表示する機能にする.\textcircled{\scriptsize{3}}「主張と根拠」では「問いと答え」から作り出した文章が主張に沿っているか確認をしながら文章作成する必要があることや主張に対した根拠は複数必要であり,参考文献として引用した論文や本のタイトルなど記録しておくことが必要とされる.そのため主張と根拠は別に表示できるようにし,根拠と参考文献を関連させるような機能にする.また実際書く文章は別のアプリケーションで記述することにする.


\begin{figure}[h]
\begin{center}
 \includegraphics[clip,width=85mm,height=55mm]{pp09.pdf}
\end{center}
 \caption{ツールの構成}
 \label{fig:教科書}
\end{figure}


\section{今後の予定}
実際に論文的な文章作成支援ツールを開発し,レポートや論文などを書いてもらい有用性を検証する.

\begin{thebibliography}{99}
\bibitem{ren00} 文部科学省:"高等学校学習指導要領",
\url{http://www.mext.go.jp/component/a_menu/education/micro_detail/__icsFiles/afieldfile/2018/07/11/1384661_6_1_2.pdf}
\bibitem{ren01} 堀 一成,坂尻 彰宏:"阪大生のためのアカデミックライティング",
\url{https://ir.library.osaka-u.ac.jp/repo/ouka/all/27153/Academic%20Writing%20Introduction.pdf}, 2019/8/23参照


\end{thebibliography}

\end{document}